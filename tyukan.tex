% !TeX program = lualatex
% !TeX encoding = UTF-8
\documentclass[11pt, aspectratio=169]{beamer}

% --- パッケージ設定 ---
\usepackage{luatexja}
\usepackage[haranoaji]{luatexja-preset} % 日本語フォント設定
\renewcommand{\kanjifamilydefault}{\gtdefault} % ゴシック体標準
\usepackage{graphicx}     % 画像挿入
\usepackage{booktabs}     % 綺麗な表
\usepackage{tikz}         % 図形描画

% --- テーマ設定 ---
% --- テーマ設定 ---
\usetheme{Madrid}
% 和やかな配色の定義 (Matcha Green & Earthy Brown)
\definecolor{MatchaGreen}{RGB}{119, 158, 119} % 抹茶色
\definecolor{Cream}{RGB}{250, 248, 240}       % クリーム色
\definecolor{DarkBrown}{RGB}{80, 50, 40}      % 焦げ茶色

\usecolortheme[named=MatchaGreen]{structure}
\setbeamercolor{palette primary}{bg=MatchaGreen, fg=white}
\setbeamercolor{palette secondary}{bg=MatchaGreen!80!white, fg=white}
\setbeamercolor{palette tertiary}{bg=DarkBrown, fg=white}
\setbeamercolor{frametitle}{bg=MatchaGreen, fg=white}
\setbeamercolor{title}{bg=DarkBrown, fg=white}
\setbeamercolor{background canvas}{bg=Cream} % 背景をクリーム色に

\usefonttheme{professionalfonts} % 数式フォントを美しく

% --- 表紙情報 ---
\title[MLB投球分類と誤分類解析]{MLB Statcastデータを用いた\\投球タイプ分類と誤分類の構造解析}
\subtitle{〜カットボール(FC)の識別精度向上に向けて〜}
\author[浅尾・川野辺・花井]{浅尾 友志, 川野辺 旭, 花井 龍悟}
\institute[浜田研]{浜田研究室}
\date{2025/12/19}

\begin{document}

% --- P1. 表紙 ---
\begin{frame}
  \titlepage
\end{frame}

% --- P2. 背景と目的 ---
\begin{frame}{背景と目的}
  \begin{block}{背景: 現代野球におけるデータ活用}
    \begin{itemize}
      \item MLBでは全投球の物理データ(Statcast)が計測されている。
      \item 速度、回転数、変化量などから球種を自動判定する需要が高い。
    \end{itemize}
  \end{block}

  \begin{alertblock}{本研究の目的}
    \begin{enumerate}
      \item 機械学習(アンサンブル学習)を用いた高精度な分類器の構築
      \item \textbf{誤分類(Error Analysis)を通じた「人間に近い識別」の実現}
      \item 特に判別が困難な\textbf{「カットボール(FC)」}の精度向上
    \end{enumerate}
  \end{alertblock}
\end{frame}

% --- P3. 実装パイプライン ---
\begin{frame}{実装パイプライン}
  一般的なフローに加え、\textbf{分析フェーズ}を重視したサイクルを構築。

  \vspace{1em}
  \centering
  \begin{tikzpicture}[node distance=3cm, auto, >=stealth]
    % ノード定義
    \node[draw, rectangle, rounded corners] (data) {データ取得};
    \node[draw, rectangle, rounded corners, right of=data] (prep) {前処理};
    \node[draw, rectangle, rounded corners, right of=prep] (model) {モデル構築};
    \node[draw, rectangle, rounded corners, fill=yellow!20, right of=model, align=center] (analysis) {\textbf{Error Analysis}\\(本日の焦点)};
    
    % 矢印
    \draw[->, thick] (data) -- (prep);
    \draw[->, thick] (prep) -- (model);
    \draw[->, thick] (model) -- (analysis);
    \draw[->, thick, dashed, bend left] (analysis) to node[above] {Feature Engineering} (prep);
  \end{tikzpicture}

  \vspace{1em}
  \begin{itemize}
    \item \textbf{ライブラリ:} \texttt{pybaseball}, \texttt{scikit-learn}, \texttt{LightGBM} 等
    \item \textbf{モデル:} Random Forest, XGBoost, LightGBMのアンサンブル
  \end{itemize}
\end{frame}

% --- P4. ベースライン評価 ---
\begin{frame}{モデル構築とベースライン評価}
  \begin{columns}
    \begin{column}{0.5\textwidth}
      \textbf{現状のスコア:}
      \begin{itemize}
        \item Accuracy: \textbf{83.0\%}
        \item Macro F1: 78.5\%
      \end{itemize}
      
      \vspace{1em}
      \textbf{課題:}
      \begin{itemize}
        \item 全体的には良好だが、特定の球種で迷いが見られる。
        \item 特に\textbf{FC(カットボール)}のRecallが低い。
      \end{itemize}
    \end{column}
    
    \begin{column}{0.5\textwidth}
      \centering
      % ここにPythonで作った Classification Report のスクショなどを貼る
      % \includegraphics[width=\textwidth]{classification_report.png}
      \fbox{\begin{minipage}{0.9\textwidth}
        \centering
        \vspace{2cm} Classification Report \\ (画像プレースホルダー) \vspace{2cm}
      \end{minipage}}
    \end{column}
  \end{columns}
\end{frame}

% --- P5. 課題の発見 (ここからデモへ繋ぐ) ---
\begin{frame}{課題の発見:カットボール(FC)の壁}
  混同行列(Confusion Matrix)による詳細分析。
  
  \begin{columns}
    \begin{column}{0.6\textwidth}
      \centering
      % ここにPythonで作った Confusion Matrix の画像を貼る
      % \includegraphics[width=\textwidth]{confusion_matrix.png}
      \fbox{\begin{minipage}{0.9\textwidth}
        \centering
        \vspace{2.5cm} Confusion Matrix \\ (FCの誤分類を示す図) \vspace{2.5cm}
      \end{minipage}}
    \end{column}
    \begin{column}{0.4\textwidth}
      \begin{itemize}
        \item FCが\textbf{SL(スライダー)}および\textbf{FF(ストレート)}と混同されている。
        \item 単純な「変化量」や「球速」だけでは分離できていない可能性。
      \end{itemize}
    \end{column}
  \end{columns}
\end{frame}

% --- P6. 仮説と検証 ---
\begin{frame}{仮説と検証:なぜ間違えるのか?}
  \begin{block}{仮説}
    「FCとSLは、変化量($pfx\_x, pfx\_z$)の分布において重なりが大きく、モデルが境界線を引けていないのではないか?」
  \end{block}

  \vspace{0.5em}
  \textbf{検証(物理的特徴量の可視化):}
  \begin{itemize}
    \item 以下の散布図において、赤点(誤分類)が緑点(正解)と完全に重複している。
    \item \textbf{結論:} 既存の特徴量(絶対値)だけでは限界がある。
  \end{itemize}
  
  % ここにPythonで作った 散布図(plot_misclassification) の画像を貼る
  % \centering \includegraphics[height=3.5cm]{scatter_plot.png}
\end{frame}

% --- P7. 改善へのアプローチ ---
\begin{frame}{改善へのアプローチ:Feature Engineering}
  ドメイン知識に基づき、モデルに「コンテキスト」を与える。

  \begin{enumerate}
    \item \textbf{相対速度 (Velocity Diff):}
      \begin{itemize}
        \item その投手の「平均ストレート球速」との差分をとる。
        \item 投手ごとの球速差をキャンセルし、球種の定義を明確化。
      \end{itemize}
      
    \item \textbf{回転効率 (Spin Efficiency) の推定:}
      \begin{itemize}
        \item 「回転数が多いのに変化しない(ジャイロ成分)」をFCの特徴として捉える。
        \item $ \frac{\text{実際の変化量}}{\text{回転数から予測される理論最大変化量}} $
      \end{itemize}
  \end{enumerate}
\end{frame}

% --- P8. 今後の展望 ---
\begin{frame}{まとめと今後の展望}
  \begin{itemize}
    \item \textbf{まとめ:}
      \begin{itemize}
        \item MLB投球データの分類モデルを構築。
        \item Error Analysisにより、FCの誤分類原因を物理的に特定。
      \end{itemize}
      
    \item \textbf{今後の展望:}
      \begin{itemize}
        \item 提案した新特徴量(相対速度、回転効率)の実装と再学習。
        \item デモ用分析ツールのUI改良。
      \end{itemize}
  \end{itemize}
  
  \vspace{1em}
  \centering
  \textbf{この後、実際のモデルの動作と分析の様子をデモでお見せします。}
\end{frame}

\end{document}